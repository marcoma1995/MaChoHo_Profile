\documentclass[%
a4paper,            % you can select other formats
11pt,               % font size(12pt, 11pt (Standard), ...)
bibliography=totoc, % bibliography in table of contents
]
{scrartcl}

\usepackage{geometry}   % control side spacing
\usepackage{setspace}   % control line spacing
\usepackage{hyperref}   % links
\usepackage{booktabs}       % nicer looking tables
\usepackage{times}
\usepackage[utf8]{inputenc}
\usepackage{textgreek}

\usepackage{graphicx}
\geometry{left=30mm, right=20mm, top=20mm, bottom=20mm, footskip=10mm} % settings for content part

\begin{document}
	\setcounter{secnumdepth}{0}
	\section{Parenting and the formation of childrens' personality traits}
	
	
	In the analysis, we focus on how parental involvement is shaping personality characteristics, such as locus of control, conscientiousness and neuroticism, which are strong predictors of educational and labour market success. We assume that parents know what is best for their children and investigate whether parents are able to shape this traits in their children through more involvement, even without displaying these personality themselves.\\\newline
	The analysis results indicate that a stronger involvement of parents in the children's life has a positive effect on the personality traits that predict better outcomes. It also suggest that mothers and fathers might have different roles in the formation of the child's personality. We find that the mother's involvement strongly increases the locus of control trait while the involvement of the fathers increases level of conscientiousnes.\\\newline
	The data used in the research mainly obtained from the The German Socio-Economic Panel (SOEP). The motivation, methodology, data description and the result of the research will included in the presentation. \\\newline
	Full coding and documentation can retrited from the following links:\\\newline  
	\url{https://github.com/marcoma1995/parenting_and_personality.git}\
	
\end{document}
